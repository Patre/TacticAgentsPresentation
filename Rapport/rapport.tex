
\PassOptionsToPackage{table}{xcolor}
\documentclass{article}

\usepackage[francais]{babel}
\usepackage[T1]{fontenc}
\usepackage[utf8]{inputenc}
\usepackage{xcolor}
\usepackage{graphicx}
\usepackage[colorlinks=true]{hyperref}
\hypersetup{urlcolor=blue,linkcolor=red,colorlinks=false} 
\usepackage{algorithm,algorithmic}
\usepackage{listings}
\usepackage{amssymb}
\usepackage{setspace}
\usepackage{listings}
\usepackage{lscape}
\usepackage{amsmath}
\usepackage[margin=2.5cm]{geometry}


% Romain
\newcommand{\cRM}[1]{\MakeUppercase{\romannumeral #1}}  % Capital
\newcommand{\cRm}[1]{\textsc{\romannumeral #1}} % Petit majuscule
\newcommand{\crm}[1]{\romannumeral #1}
% Siècle %
\newcommand{\siecle}[1]{\cRM{#1}\textsuperscript{e}~siècle}


\definecolor{keywords}{RGB}{255,0,0}
\lstset{language=[LaTeX]TeX,
texcsstyle=*\color{keywords},
breaklines=true,
keywordstyle=\color{keywords},
commentstyle=\color{darkgreen},
tabsize=2,
backgroundcolor=\color{lightgrey},
escapeinside=||,
morekeywords={*,subsection,make title,tableofcontents,include graphics}
}


\rowcolors{1}{gray!25}{white}

\title{Les stratégies militaires dans les Systèmes Multi-Agents}
\author{Chloé Desdouits, William Dyce}
\date{\today}


\begin{document}


\maketitle

\tableofcontents
\newpage


\section{Organisations guerrières}
Il y a deux types d'organisations militaires bien différents : les armées et les guérillas. Nous allons présenter pour chaque type sa structure et le contexte dans lequel il existe.

\subsection{Armées}
Les armées sont des entités qui tirent leur légitimité de leur appartenance aux états ou jadis aux institutions religieuses.

\subsubsection{Structure}
La structure des armées est homogène : en effet, la hiérarchie est sous forme d'arbre dont les nœuds du même niveau ont le même nombre de fils.
\begin{figure}[H]
	\begin{centering}
	\includegraphics[width=0.8\linewidth]{../ressources/armee_cesar}
	\caption{Exemple d'organisation homogène de l'armée (armée de César - Warmaster ancients \cite{armee_de_cesar})}
	\end{centering}
\end{figure}

Cela permet de construire des formations à grande échelle qui aient une structure prédéfinie.
\begin{figure}[H]
\begin{centering}
\begin{tabular}{| c l | c l | c l | c l |}
	\hline
	\multicolumn{8}{| c |}{\textbf{Armée}}
	\\
	\multicolumn{2}{| c |}{\textbf{Spartiate}} 	& \multicolumn{2}{ c |}{\textbf{Romaine}} & \multicolumn{2}{ c |}{\textbf{Perse}}	& \multicolumn{2}{ c |}{\textbf{Mongole}} 	\\
	\hline
	 					&			& 					&			& 				&			& \textit{Ordu}		& > 10000			\\
	 \itshape Mora 			& 576		& \itshape Légion		& 6000		& \textit{Baivarabam}& 10000		& \textit{Tumen} 	& 10000			\\
	 \itshape Loche			& 144		& \itshape Cohorte		& 600		& \textit{Hazarabam}	& 1000		& \textit{Minghan}  	& 1000			\\
	 \itshape Pentécostye	& 72			& \itshape Manipule		& 200		& \textit{satabam}	& 100		& \textit{Zuut} 		& 100			\\
	 \itshape Énomotie		& 36			& \itshape Centurie		& 100		& \textit{Dathabam} 	& 10			& \textit{Arav} 		& 10				\\
	\hline
\end{tabular}
\caption{Effectif des unités dans différentes armées antiques \cite{mongol_army, spart_army, roman_legion,persian_army}}
\end{centering}
\end{figure}


\subsubsection{Contexte}
Les armées interviennent dans deux contextes principaux : la défense de la patrie et la conquête de nouveaux territoires.
\begin{figure}[H]
	\begin{centering}
	\includegraphics[width=\linewidth]{../ressources/dmz-corea}
	\caption{Zone démilitarisée entre la Corée du Sud et la Corée du Nord \cite{dmz_corea}}
	\end{centering}
\end{figure}
\begin{figure}[H]
	\begin{centering}
	\includegraphics[]{../ressources/risk-board-game}
	\caption{Tentative de conquête dans le jeu Risk \cite{risk_picture}}
	\end{centering}
\end{figure}




\subsection{Guérillas}
Les guérillas se forment à la suite d'une scission idéologique entre le pouvoir en place et une partie de la population. Elles ont souvent à leur tête un leader charismatique.

\subsubsection{Structure}
Ces organisation sont souvent structurées de façon très hétérogène : peu de responsables et des formations à petite échelle.
\begin{figure}[H]
	\begin{centering}
	\includegraphics[width=\linewidth]{../ressources/guerrilla_naxalite}
	\caption{Guérilla Naxalite (Inde) \cite{guerrilla_naxalite}}
	\end{centering}
\end{figure}

Les guérillas, comme le montre la figure suivante, ont une échelle réduite par rapport aux forces contre lesquelles elles luttent.
\begin{figure}[H]
	\begin{centering}
	\begin{tabular}{| l | c | c | c | c |}
	\hline
	\textbf{Localisation}		& \textbf{Irlande} 	& \textbf{Inde} 	& \textbf{Cuba}	& \textbf{Colombie}	\\
	\hline
	\textbf{Régime}			& 55.000			& 1.414.000	& 35.000		& 300.000			\\
	\textbf{Insurgés}		& 15.000			& 15.000		& 200		& 20.000			\\
	\hline
	\end{tabular}
	\caption{Effectifs des forces du régime par rapport aux forces des guérillas \cite{naxalite_guerrilla_wiki, irish_civil_war_wiki}}
	\end{centering}
\end{figure}


\subsubsection{Contexte}
Les forces de guérilla sont adaptées au terrain : montagnes, maquis, villes \ldots
\begin{figure}[H]
	\begin{centering}
	\includegraphics[width=\linewidth]{../ressources/valdepenas}
	\caption{Soulèvement de Valdepeñas ; guérilla urbaine \cite{valdepenas}}
	\end{centering}
\end{figure}
Elles sont mobiles.
\begin{figure}[H]
	\begin{centering}
	\includegraphics[]{../ressources/guerrilla_colombia}
	\caption{Guérilla colombienne \cite{guerrilla_colombia}}
	\end{centering}
\end{figure}
La plupart des recrues sont des civils.
\begin{figure}[H]
	\begin{centering}
	\includegraphics[]{../ressources/rebel_syrie}
	\caption{Des rebelles en Syrie \cite{rebel_syrie}}
	\end{centering}
\end{figure}


\section{Stratégies militaires}

\subsection{Définitions}
Politique \cite{politique_jomini}
\begin{quote}“La politique de la guerre c’est tout simplement décider où, quand, comment, avec quels alliés et pourquoi entrer en guerre.”\end{quote}
\begin{figure}[H]
	\begin{centering}
	\includegraphics[width=0.94\linewidth]{../ressources/alliances_ww1}
	\caption{Alliances militaires en Europe 1914-1918 \cite{ww1}}
	\end{centering}
\end{figure}

Stratégie \cite{military_strategy}
\begin{quote}“La stratégie militaire est l'art de coordonner -au plus haut niveau de décision- l'action de l'ensemble des forces militaires de la Nation pour conduire une guerre, gérer une crise ou préserver la paix.”\end{quote}
\begin{figure}[H]
	\begin{centering}
	\includegraphics[width=\linewidth]{../ressources/strategy_ww1}
	\caption{Plans de bataille des états-majors allemand et français pendant la WW1 \cite{ww1}}
	\end{centering}
\end{figure}

Tactique \cite{tactic}
\begin{quote}“La tactique est l'art de diriger une bataille, en combinant, par la manœuvre, l'action des différents moyens de combat en vue d'obtenir le maximum d'efficacité.”\end{quote}
\begin{figure}[H]
	\begin{centering}
	\includegraphics[width=\linewidth]{../ressources/Battles_of_Charleroi_ww1}
	\caption{Mouvement de troupes pendant la bataille de Charleroi (WW1) \cite{charleroi_battle}}
	\end{centering}
\end{figure}



\subsection{Stratégies}

\subsubsection{Sun Tzu}
Chine (\siecle{6} BC) \hfill \begin{minipage}{5cm}
\includegraphics[width=\linewidth]{../ressources/sun_tzu_general}
\end{minipage}
\begin{quote}“L'art de la guerre, c'est de soumettre l'ennemi sans combat.”\end{quote}

Préceptes
\begin{itemize}
\item prendre toutes les possessions de l'adversaire et les conserver intactes
\item adaptabilité, préparation, connaissance du terrain et des forces en présence (espionnage)
\end{itemize}
Axes stratégiques
\begin{enumerate}
\item cause morale
\item conditions climatiques
\item conditions géographiques
\item qualités du dirigeant
\item organisation et discipline
\end{enumerate}

\cite{tzu1997art, sun_tzu_fighting, sun_tzu_wiki}

\subsubsection{Alexandre le grand}
Grèce (\siecle{4} BC)
\hfill \begin{minipage}{5cm}
\includegraphics[width=\linewidth]{../ressources/AlexanderTheGreat_Bust}
\end{minipage}
\begin{quote}“Ce qui ne me tue pas me rend plus fort.”\end{quote}

Préceptes
\begin{itemize}
\item conscription et intégration des peuples vaincus
\item allègement de l'équipement des troupes
\end{itemize}
Axes stratégiques
\begin{enumerate}
\item assurer ses arrières
\item choisir judicieusement la voie d'accès pour chaque conquête
\end{enumerate}
\cite{alexander_the_great, alexandre_balkans}

\subsubsection{Julius Caesar}
Italie (\siecle{1} BC)
\hfill \begin{minipage}{5cm}
\includegraphics[width=\linewidth]{../ressources/cesare}
\end{minipage}
\begin{quote}“L’expérience, voilà le maître en toutes choses.”\end{quote}

Préceptes
\begin{itemize}
\item stabilité militaire et logistique
\end{itemize}
Axes stratégiques
\begin{enumerate}
\item infanterie lourde
\item bataillons étrangers spécialisés
\item formations en fonction des conditions géographiques
\item bivouac fortifié
\end{enumerate}
\cite{caesar_wiki, caesar_lacks}

\subsubsection{Genghis Khan}
Mongolie (\siecle{12})
\hfill \begin{minipage}{5cm}
\includegraphics[width=\linewidth]{../ressources/genghis_khan}
\end{minipage}
\begin{quote}“Le plus grand bonheur du Mongol est de vaincre l’ennemi, de ravir ses trésors, de faire hurler ses serviteurs, de se sauver au galop de ses chevaux bien nourris [\ldots]”\end{quote}

Préceptes
\begin{itemize}
\item guerre psychologique
\item règne de la terreur
\item connaissance du terrain : espionnage ; éclaireurs
\end{itemize}
Axes stratégiques
\begin{enumerate}
\item peu de troupes ; avant-garde forte
\item troupes montées % logistique et vitesse
\item délégation des décisions
\item relais de communication et ravitaillement
\item attaques biologiques
\end{enumerate}
\cite{khan_wiki, military_strategy, mongol_army}

\subsubsection{Napoléon Bonaparte}
France (\siecle{18})
\hfill \begin{minipage}{5cm}
\includegraphics[width=\linewidth]{../ressources/napoleon}
\end{minipage}
\begin{quote}“Réunir ses feux contre un seul point ; une fois la brèche faite, l’équilibre est rompu, tout le reste devient inutile.”\end{quote}
Préceptes
\begin{itemize}
\item recherche systématique de la bataille
\item destruction totale des forces adverses
\item être le plus fort à l’endroit où l’on a décidé de frapper le coup décisif
\end{itemize}
Axes stratégiques
\begin{enumerate}
\item vitesse de manœuvre : {\emph Blitzkrieg}
\item fortifications
\item lignes de réapprovisionnement provisoires
\item artillerie
\end{enumerate}
\cite{napoleon, napoleon_wiki, napoleon_portrait}


\subsubsection{Conclusion}
Nous pouvons donc déduire des tous ces exemples deux types distincts de stratégies : les stratégies indirectes et les stratégies directes.

\bigskip
\begin{tabular}{|p{0.45\linewidth}|p{0.45\linewidth}|}
\hline
\emph{Stratégie indirecte} & \emph{Stratégie directe}\\
\hline
renseignement & conscription\\
embuscade & recherche de la bataille décisive\\
tromperie & planification et formations\\
sabotage & fortifications\\
\hline
\end{tabular}

\bigskip
Deux exemples de stratégies directes :
\begin{figure}[H]
	\begin{centering}
	\includegraphics[width=0.8\linewidth]{../ressources/Vauban_Fort_Lupin}
	\caption{Vue aérienne du Fort Lupin construit par Vauban (Rochefort). \cite{fort_lupin}}
	\end{centering}
\end{figure}
\begin{figure}[H]
	\begin{centering}
	\includegraphics[width=\linewidth]{../ressources/tortue}
	\caption{La tortue : formation défensive romaine. \cite{turtle_form}}
	\end{centering}
\end{figure}

Deux exemples de stratégies indirectes :
\begin{figure}[H]
	\begin{centering}
	\includegraphics[width=0.8\linewidth]{../ressources/mongol_army}
	\caption{Charge de cavalerie mongole. \cite{mongol_cavalery}}
	\end{centering}
\end{figure}
\begin{figure}[H]
	\begin{centering}
	\includegraphics[width=0.8\linewidth]{../ressources/sabotage_maquisards}
	\caption{Sabotage d'une voie de chemin de fer - guerre de 1914. \cite{sabotage}}
	\end{centering}
\end{figure}
\cite{war}


\subsection{Formations et unités}

\subsubsection{Infanterie}
L'infanterie est la matière première de toute armée et existe toujours en tant qu'unité. Avant l'arrivée des armes à feu l'infanterie fut utilisée comme un mur de chair mobile ayant appris à marcher en un bloc contigüe afin que chaque homme puisse défendre l'homme à sa gauche.

\begin{center}
\begin{figure}[H]
\hfill
\begin{minipage}[H]{0.3\linewidth}
	\centering
	\includegraphics[width=\linewidth]{../ressources/Roman_soldier}
	\caption{Uniforme d'un légionnaire romain du \siecle{1} siècle. \cite{infantery}}
\end{minipage}
\hfill
\begin{minipage}[H]{0.6\linewidth}
	\centering
	\includegraphics[width=\linewidth]{../ressources/JGDSF_Soldiers}
	\caption{Japan Ground Self-Defense Force infantry. \cite{infantery}}
\end{minipage}
\hfill
\end{figure}
\end{center}

Dans l'époque moderne l'infanterie est devenue une forme d'unité plutôt auxiliaire. Elle se justifie encore par sa relative mobilité (à travers un terrain difficile) par rapport aux véhicules.

\subsubsection{Cavalerie légère}
Grâce à sa mobilité, la cavalerie légère fait de bons éclaireurs, tirailleurs et saboteurs. Elle sert surtout à nuire à la logistique et à la discipline de l'armée adverse en s'en prenant à des cibles plus faibles, mais aura du mal à attaquer directement des masses d'infanterie lourde.
\begin{center}
\begin{figure}[H]
\hfill
\begin{minipage}[H]{0.45\linewidth}
	\centering
	\includegraphics[width=\linewidth]{../ressources/IlkhanidHorseArcher}
	\caption{Archer à cheval Houlagides. \cite{archery}}
\end{minipage}
\hfill
\begin{minipage}[H]{0.4\linewidth}
	\centering
	\includegraphics[width=\linewidth]{../ressources/MongolCavalrymen}
	\caption{Archers mongoles à cheval au combat (illustration d'un manuscrit du début du xive siècle) \cite{mongol_army}}
\end{minipage}
\hfill
\end{figure}
\end{center}
La cavalerie légère n'existe peut-être plus dans l'époque moderne, mais ses tactiques sournoises persistent à travers les guérillas et commandos, par exemple.


\subsubsection{Cavalerie lourde}
La cavalerie lourde, inversement à la cavalerie légère, sert de marteau pour briser les masses d'infanterie. C'est une \og{}troupe de choc\fg{} antique, souvent gardée en réserve et engagée seulement au moment critique et à l'endroit critique.
\begin{center}
\begin{figure}[H]
\hfill
\begin{minipage}[H]{0.4\linewidth}
	\centering
	\includegraphics[width=\linewidth]{../ressources/Ottoman_Mamluk_horseman}
	\caption{Ottoman Mamluk heavy cavalry, circa 1550. \cite{heavy_cavalry}}
\end{minipage}
\hfill
\begin{minipage}[H]{0.45\linewidth}
	\centering
	\includegraphics[width=\linewidth]{../ressources/cuirassiers}
	\caption{French cuirassiers, 19th century \cite{heavy_cavalry}}
\end{minipage}
\hfill
\end{figure}
\end{center}
Si la cavalerie lourde n'existe plus, l'utilisation de tanks et de bombardements (Blitzkrieg) concentrés en un point de la ligne peut être vu comme une continuation de la stratégie de choc.

\subsubsection{Formation en triple ligne}
La tactique consistant de garder ses meilleures troupes à l'arrière est illustrée par la disposition en triple ligne de l'armée Romaine : la première ligne est formée de soldats débutants (Hastati), la deuxième du corps principal de soldats plus expérimentés (Principes), et la troisième des vétérans (Triarii). Ces lignes de défense supplémentaires servent à combler les trous qui pourraient se former dans la première. 
\begin{figure}[H]
	\begin{centering}
	\includegraphics[width=\linewidth]{../ressources/Polybian_formation}
	\caption{Disposition classique en trois lignes \cite{roman_infantry_tactics}}
	\end{centering}
\end{figure}
La disposition indentée permet en plus aux tirailleurs de se retirer à travers les lignes de l'armée sans semer panique et confusion. 
\begin{figure}[H]
	\begin{centering}
	\includegraphics[width=\linewidth]{../ressources/Formations_infanterie_romaine}
	\caption{Formations alternatives \cite{roman_infantry_tactics}}
	\end{centering}
\end{figure}

\subsection{Tactiques}

\subsubsection{Charge}
La charge consiste en une attaque à toute vitesse qui se concentre généralement sur un point fixe, souvent pour profiter d'un instant de faiblesse opportun. 

\begin{center}
\begin{figure}[H]
\hfill
\begin{minipage}[H]{0.5\linewidth}
	\centering
	\includegraphics[width=\linewidth]{../ressources/Charge-de-cavalerie}
	\caption{Charge de cavalerie de la garde républicaine \cite{charge_cavalery}}
\end{minipage}
\hfill
\begin{minipage}[H]{0.45\linewidth}
	\centering
	\includegraphics[width=\linewidth]{../ressources/charge_infanterie}
	\caption{Charge d’infanterie française 1914 \cite{infantery_charge,charge_tactic}}
\end{minipage}
\hfill
\end{figure}
\end{center}
La charge est une tactique visant surtout à faire peur et donc à dissoudre l'autre armée. Une défense enracinée qui garde sa discipline arrivera très probablement à se défendre contre une charge, surtout une défense enracinée munie de mitrailleuses. Les scènes de boucherie lors de la Première Guerre Mondiale ont mis fin à la notion de charge glorieuse.

\subsubsection{Manœuvre de flanquement}
Avant l'arrivée des armes à feu, un bloc d'infanterie était particulièrement vulnérable à droite, le bouclier étant porté dans la main gauche. En plus de cette mobilité, une attaque sur le flanc d'une formation réduit gravement sa mobilité et peut avoir de forts impacts psychologiques \cite{tactic,pincer_tactic}.
\begin{center}
\begin{figure}[H]
\hfill
\begin{minipage}[H]{0.5\linewidth}
	\centering
	\includegraphics[width=\linewidth]{../ressources/Battle_of_Marathon}
	\caption{Bataille de Marathon (double enveloppement, manœuvre de flanquement) \cite{flanking_maneuver}}
\end{minipage}
\hfill
\begin{minipage}[H]{0.45\linewidth}
	\centering
	\includegraphics[width=\linewidth]{../ressources/Battle_of_Mohi}
	\caption{Plan de la bataille de Mohi (Hongrie) \cite{mohi_battle}}
\end{minipage}
\hfill
\end{figure}
\end{center}

\subsubsection{Le marteau et l'enclume}
Cette tactique fameuse d'Alexandre le Grand se déroule selon deux étapes. Dans un premier temps la cavalerie Macédonienne contourne l'armée adverse pour l'attaquer par l'arrière :
\begin{center}
\begin{figure}[H]
\hfill
\begin{minipage}[H]{0.5\linewidth}
	\centering
	\includegraphics[width=\linewidth]{../ressources/marteau}
\end{minipage}
\hfill
\begin{minipage}[H]{0.45\linewidth}
	\centering
	\includegraphics[width=\linewidth]{../ressources/marteau2}
\end{minipage}
\hfill
\caption{Le \og{}marteau\fg{} d'Alexandre le Grand}
\end{figure}
Cette attaque sert à distraire et à embrouiller l'ennemi et avant qu'il puisse se reformer, la force principale d'Alexandre arrive par devant \cite{Alexanders_tactics}.
\begin{figure}[H]
\hfill
\begin{minipage}[H]{0.5\linewidth}
	\centering
	\includegraphics[width=\linewidth]{../ressources/enclume}
\end{minipage}
\hfill
\begin{minipage}[H]{0.45\linewidth}
	\centering
	\includegraphics[width=\linewidth]{../ressources/enclume2}
\end{minipage}
\hfill
\caption{\og{}L'enclume\fg{} d'Alexandre le Grand \cite{Alexanders_tactics}}
\end{figure}
\end{center}
Les forces adverses se trouvent alors entourées de tous les côtés et donc incapables de focaliser sur une contre-attaque.

\subsubsection{Contre-attaque}
Une avancée faite de manière trop impétueuse peut laisser l'attaquant fatigué, désorganisé et loin de ses soutients, donc vulnérable à ce que nous appelons \og{}contre-attaque\fg{}.
\begin{center}
\begin{figure}[H]
\hfill
\begin{minipage}[H]{0.5\linewidth}
	\centering
	\includegraphics[width=\linewidth]{../ressources/Cedar_Creek_Confederate_attacks}
	\caption{Battle of Cedar Creek, Confederate attacks}
\end{minipage}
\hfill
\begin{minipage}[H]{0.45\linewidth}
	\centering
	\includegraphics[width=\linewidth]{../ressources/Cedar_Creek_Union_counterattack}
	\caption{Battle of Cedar Creek, Union counterattack}
\end{minipage}
\hfill
\end{figure}
\cite{counterattack_wiki, couterattack_cedar_creek}
\end{center}

\subsubsection{Hit and Run et Retraite feinte}
Une déroute simulée peut aspirer l'attaquant dans un piège, laissant au défenseur la possibilité de lancer une contre-attaque pendant ce moment de faiblesse.
\begin{figure}[H]
	\begin{centering}
	\includegraphics[width=\linewidth]{../ressources/infantrysquare3}
	\caption{These French troops broke formation, and are pursuing the enemy cavalry. \cite{feigned_retreat}}
	\end{centering}
\end{figure}
C'est la tactique préférée des unités mobiles munies d'armes de distance, surtout la cavalerie légère des Steppes (dont les Mongoles). Les missiles lancés par ces tirailleurs servent à provoquer l'ennemi à les suivre ou au moins de se mettre en une formation espacée moins favorable contre une attaque au corps-à-corps. Nous parlons alors de tactique "Hit and Run" \cite{mongol_army}.


\subsubsection{Embuscade}
Une armée se déplace en général selon une formation globale qui favorise la vitesse au dépend de l'efficacité au combat (par exemple en colonne), donc a besoin de se déployer avant de se battre.
\begin{figure}[H]
	\begin{centering}
	\includegraphics[width=\linewidth]{../ressources/Uzbin_valley_ambush-map}
	\caption{Embuscade d'Uzbin \cite{uzbin_ambush}}
	\end{centering}
\end{figure}
De ce fait une attaque surprise, en plus de ses impacts psychologiques, permet à l'attaquant de profiter de la faiblesse relative du défenseur.
\begin{figure}[H]
	\begin{centering}
	\includegraphics[width=0.8\linewidth]{../ressources/ambush}
	\caption{Malaysian Armed Forces ambush \cite{ambush_picture}}
	\end{centering}
\end{figure}
\cite{ambush_wiki}

\section{Applications aux systèmes multi-agents}

\subsection{Enhanced Isaac Neural Simulation Toolkit (EINSTein)}
%http://wwwhome.ewi.utwente.nl/~hmiproj6/AL/isaac_einstein_paper.pdf
%https://hmi.ewi.utwente.nl/~hmiproj6/AL/mors.pdf
%http://en.wikipedia.org/wiki/Braitenberg_vehicle
%http://en.wikipedia.org/wiki/Lanchester's_laws
%http://www2.dcs.hull.ac.uk/NEAT/dnd/AI/AndyIlachinski_BriefingSlides.pdf

\subsubsection{Origines}
\textbf{EINSTein} (\underline{E}nhanced \underline{I}SAAC \underline{N}eural \underline{S}imulation \underline{T}oolkit), conçu sous Andy Ilachinksi en 1999, se présente comme \og{}laboratoire\fg{} de vie artificielle, fut implémenté dans le but de faciliter l'exploration de l'auto-organisation et l'émergence dans le combat terrestre. 

\begin{center}
\begin{figure}[H]
\begin{minipage}[H]{0.25\linewidth}
	\centering
	\includegraphics[width=\textwidth]{../ressources/lanchester}
	\caption{Frederick W. Lanchester}
\end{minipage}
\hfill
\begin{minipage}[H]{0.25\linewidth}
	\centering
	\includegraphics[width=\textwidth]{../ressources/valentin_braitenberg}
	\caption{Valentin Braintenberg}
\end{minipage}
\hfill
\begin{minipage}[H]{0.25\linewidth}
	\centering
	\includegraphics[width=\textwidth]{../ressources/ilachinski}
	\caption{Andy Ilachinski}
\end{minipage}
\end{figure}
\end{center}

Il se base sur \textbf{ISAAC} (\underline{I}rreducible \underline{S}emi-\underline{A}utonomus \underline{A}daptive \underline{C}ombat)~: modèle créé par opposition à la vision réductionniste et par le haut des Lois de Lanchester, formules utilisées couramment pour modéliser l'attrition depuis 1916. ISAAC se base d'ailleurs lui-même sur les \og{}Vehicles\fg{} de Valentino Braitenberg, agents munis de senseurs et possédant des effecteurs leur permettant de se déplacer dans leur environnement. Si les "Vehicles" apportent la notion de \emph{multum in parvo} et ISAAC applique cette vision synthésiste et par le bas à la modélisation du combat terrestre, EINSTein étend, améliore et concrétise sa preuve de concept, et l'encapsule dans une boite à outils pour développeurs.

\subsubsection{Fonctionnement}

\begin{center}
\begin{figure}[H]
	\begin{centering}
	\includegraphics[width=0.9\linewidth]{../ressources/Einstein}
	\caption{Interface développée pour EINSTein}
	\end{centering}
\end{figure}
\cite{simu_guerre,ilachinski1994,ilachinski1999}
\end{center}

Un ensemble d'automates cellulaires mobiles divisés en deux équipes se confrontent, le but étant pour chaque équipe d'atteindre le drapeau de l'autre sans que le sien soit touché. Chaque agent dispose d'une personnalité~: un ensemble de pondérations pour chaque perception possible (ennemis ou alliés sains ou blessés, drapeaux, etc) différent pour chaque état. L'agent aura donc plus ou moins envie d'approcher la source d'une perception selon sa personnalité et son état physique.

\begin{center}
\begin{figure}[H]
\begin{minipage}[H]{0.35\linewidth}
	\centering
	\includegraphics[width=\textwidth]{../ressources/einstein_personality_weight}
\end{minipage}
\hfill
\begin{minipage}[H]{0.55\linewidth}
	\centering
	\includegraphics[width=\textwidth]{../ressources/einstein_personality_factor}
\end{minipage}
\caption{Agent EINSTein~: pondérations de comportements et meta-règles}
\end{figure}
\end{center}

En plus de sa personnalité, un ensemble de \og{}meta-règles\fg{} influencent ses décisions. Il aura par exemple envie d'avancer vers le drapeau ennemi, mais ne le fera pas sans qu'il ait autour de lui un certain nombre d'alliés.

\begin{figure}[H]
	\begin{center}
	\includegraphics[width=0.75\linewidth]{../ressources/einstein_global_behavior}
	\caption{Tactiques émmergeantes retrouvées dans ISAAC et EINSTein}
	\end{center}
\end{figure}

Avec cette architecture simple et un peu d'apprentissage automatique par algorithmes génétiques, nous arrivons à des tactiques intéressantes telles que les formations en pointe, encerclement, prises en tenaille, attaques guérilla,~\dots

\subsection{Iruba}


\begin{center}
\begin{figure}[H]
\hfill
\begin{minipage}[H]{0.25\linewidth}
	\centering
	\includegraphics[width=\textwidth]{../ressources/doran}
	\caption{Jim Doran}
\end{minipage}
\hfill
\begin{minipage}[H]{0.7\linewidth}
La simulation Iruba fut conçue en 2005 par Jim Doran pour mettre à l'épreuve un certain nombre d'hypothèses émises vis-à-vis de la guérilla. Notamment il a mis en doute les assertions du Foquisme de Che Guevera, qui suggère qu'un petit nombre d'hommes avec le support du peuple parviendra à ses fins.
Le \og{}wargame\fg{} se joue sur une grille de territoires, chacun pouvant soutenir plus ou moins les rebelles ce qui rendra plus ou moins difficile défense et recrutement chez les deux camps.
\end{minipage}
\hfill
\end{figure}
\end{center}

\begin{figure}[H]
	\begin{centering}
	\includegraphics[width=\linewidth]{../ressources/insurgency}
	\end{centering}
	\caption{Trace du déroulement d'un \og{}jeu\fg{} d'Iruba}
\end{figure}

\begin{center}
\begin{figure}[H]
\hfill
\begin{minipage}[H]{0.6\linewidth}
La simulation montre surtout l'importance de la taille initiale de la rébellion, une amorce que conduira à une boucle de rétroaction favorable pour les rebelles. Pour cette raison le gouvernement aura plus de chance de mettre en fuite les insurgés si sa réponse brise rapidement et définitivement cette boucle. Parmi les résultats on note aussi l'importance de la concentration des forces gouvernementales (importante pour elles), et de la mobilité des guérilleros (favorable seulement si elle reste modérée).
\end{minipage}
\hfill
\begin{minipage}[H]{0.3\linewidth}
\begin{algorithmic}[1]
		\WHILE{non termination}
			\STATE Attacks and their impact
			\STATE HQ decisions
			\STATE Recruitment
			\STATE Force movement
		\ENDWHILE
		\end{algorithmic}
\cite{doran2005iruba}
\end{minipage}
\hfill
\end{figure}
\end{center}
\begin{center}
\begin{figure}[H]
\begin{minipage}[H]{0.4\linewidth}
	\centering
	\includegraphics[width=\linewidth]{../ressources/iruba_counter_attack}
\end{minipage}
\hfill
\begin{minipage}[H]{0.4\linewidth}
	\centering
	\includegraphics[width=\linewidth]{../ressources/iruba_force_concentration}
\end{minipage}
\end{figure}
\begin{figure}[H]
\begin{minipage}[H]{0.4\linewidth}
	\centering
	\includegraphics[width=0.9\linewidth]{../ressources/iruba_hyper_mobility}
\end{minipage}
\hfill
\begin{minipage}[H]{0.4\linewidth}
	\centering
	\includegraphics[width=\linewidth]{../ressources/iruba_mobility}
\end{minipage}
\caption{Stratégies influant sur la réussite des agents Iruba (RFC~: concentration des forces du régime)}
\end{figure}
\end{center}

\subsection{RPDAgent (Recogition-Primed Decision Agent)}
% http://en.wikipedia.org/wiki/Recognition_primed_decision
\subsubsection{Recognition-primed Decision model}
Developpé par Gary Klein à partir de 1985, le modèle RPD (\underline{R}ecognition-\underline{P}rimed \underline{D}ecision) est un modèle théorique qui vise à décrire la manière que les êtres humains utilisent pour prendre des décisions. RPD s'inscrit dans le framework NDM (\underline{N}aturalistic \underline{D}ecision \underline{M}aking), qui s'oppose à l'agent rationnel comme modèle du décideur humain.
Un agent RPD base ses décisions sur, d'une part, une reconnaissance de motifs et de l'autre, sur un moteur de simulation~: deux mécanismes sont développés et renforcés à partir des expériences passées. Un agent RPD ne fait pas une énumération des actions possibles mais prend plutôt la première action qui semblerait adéquate d'après ce qu'il a vécu. C'est donc un agent \og{}intuitif\fg{}.

\begin{figure}[H]
\centering
\begin{minipage}[H]{0.3\linewidth}
	\centering
	\includegraphics[width=0.8\textwidth]{../ressources/klein}
	\caption{Gary Klein}
\end{minipage}
\hspace{0.1\linewidth}
\begin{minipage}[H]{0.3\linewidth}
	\centering
    \includegraphics[width=\textwidth]{../ressources/john_sokolowski}
    \caption{John A. Sokolowski}
\end{minipage}
\hfill
\end{figure}

\subsubsection{Travail de John Soklowski}
L'RPDAgent de Soklowski fut développé en 2003 dans le but de modéliser la prise de décision à grande échelle d'un général humain. Notons que la problématique n'est pas de trouver une solution optimale, mais plutôt de passer un test de Turing, l'application visée étant de simuler un adversaire réaliste lors d'un entrainement et non pas de créer une machine capable de mener des vraies opérations militaires.
\begin{figure}[H]
\centering
\begin{minipage}[H]{0.4\linewidth}
	\centering
	\includegraphics[width=\textwidth]{../ressources/RPDagent_uml}
	\caption{Diagramme UML de l'agent composite RPDAgent}
\end{minipage}
\hfill
\begin{minipage}[H]{0.55\linewidth}
En pratique l'agent RPD est un agent composite formé de plusieurs sous-agents. Il reçoit soit des appels à la decision soit des appels de révision d'actions à base de nouvelles informations, et réagit ainsi~:
\begin{algorithmic}[1]
\STATE \emph{RecognitionAgent}: analyse situation
\STATE \emph{SymbolicConstructorAgent}: create internal representation
\FORALL {decisions}
\STATE new \emph{DecisionAgent}: consider this decision
\FORALL {goals}
\STATE new \emph{ReactiveAgent}: consider this goal wrt. this decision
\ENDFOR
\ENDFOR
\end{algorithmic}
Chaque \emph{ReactiveAgent} rend une évaluation à son \emph{DecisionAgent}: s'ils sont tous d'accord, le \emph{ReactiveAgent} rend une affirmation à l'\emph{RPDAgent}, et c'est cette action qui sera choisie. À noter qu'on n'attend pas les autres \emph{DecisionAgent}~: c'est la première décision raisonnable qui est choisie.
\end{minipage}
\end{figure}
\cite{sokolowski2003}
Si jamais certains \emph{ReactiveAgent} ne sont pas satisfaits par la décision, leur \emph{DecisionAgent} leur demandera de trouver un compromis entre eux, donc de modifier la décision, et si aucun compromis ne pourra être trouvé par aucun des \emph{DecisionAgent} alors une décision par défaut sera choisie.

Un protocole expérimental montre que le RPDAgent, préparé avec une certaine base de connaissances historiques, fait des décisions semblables à celles d'un groupe de commandants humains mis devant les mêmes problèmes. 


\section{Conclusion}
\begin{figure}[H]
\includegraphics[width=\textwidth]{../ressources/donut}
\caption{Full Metal Jacket \cite{fmj_donut}}
\end{figure}
Nous remarquons les aspects intéressants des systèmes multi-agents sus-mentionnés et nous allons en tirer partie dans notre futur programme. En effet, nous allons développer les aspects antagonistes des systèmes organisés où l'individu est noyé dans le collectif et des systèmes type guérilla où les individus ont plus de degrés de liberté.

\newpage
\bibliographystyle{plain}
\bibliography{../Bib.bib}

\end{document}


