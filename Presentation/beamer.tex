
\PassOptionsToPackage{table}{xcolor}
\documentclass[utf8]{beamer}

\usepackage[T1]{fontenc}
\usepackage[french]{babel}
\usepackage{listings}
\usepackage{multicol}
\usepackage{rotating}

\setbeamertemplate{navigation symbols}{}
\usetheme{Amsterdam}

%--------------------------------------------------------------------------------
% DEFINE SECTIONPAGE
%--------------------------------------------------------------------------------
% For older version of texlive which don't support the \sectionpage command
%--------------------------------------------------------------------------------
\setbeamerfont{section title}{parent=title}
\setbeamercolor{section title}{parent=titlelike}
\defbeamertemplate*{section page}{default}[1][]
{
	\centering
		\begin{beamercolorbox}[sep=8pt,center,#1]{section title}
			\usebeamerfont{section title}\insertsection\par
		\end{beamercolorbox}
}
\newcommand*{\sectionpage}{\usebeamertemplate*{section page}}
%--------------------------------------------------------------------------------

\definecolor{keywords}{RGB}{255,0,0}
\lstset{language=[LaTeX]TeX,
texcsstyle=*\color{keywords},
breaklines=true,
keywordstyle=\color{keywords},
commentstyle=\color{darkgreen},
tabsize=2,
backgroundcolor=\color{lightgrey},
escapeinside=||,
morekeywords={*,subsection,make title,tableofcontents,include graphics}
}

%\insertframenumber/\inserttotalframenumber\hskip2ex

\definecolor{lightgrey}{rgb}{0.9,0.9,0.9}
\definecolor{darkgreen}{rgb}{0,0.6,0} 
\rowcolors{1}{gray!25}{white}

\title{Les stratégies militaires dans les Systèmes Multi-Agents}
\author{Chloé Desdouits, William Dyce}
\date{\today}

\begin{document}

\frame{\titlepage}

\begin{frame}
  \tableofcontents
\end{frame} 

\section{Organisations militaires}
\frame{\sectionpage}

\subsection{Hiérarchiques}
\begin{frame}{Structure}

Une force militaire conventionnelle est à un arbre complet où~:
\begin{itemize}
\item Chaque nœud est commandant de ses nœuds fils,
\item Les feuilles correspondent aux soldats.
\end{itemize}

\end{frame}

\begin{frame}{Structure}

\begin{center}
\begin{tabular}{ l | c c }
\cellcolor{white}	& \multicolumn{2}{c}{\textbf{Civilisation}} \\
\textbf{Effectif}	& \textbf{Perse}& \textbf{Mongol} \\
  \hline
  $>10^4$		& -										& \textit{Ordu}			\\
  $10^4$ 		& \textit{Baivarabam} & \textit{Tumen} 		\\
  $10^3$ 		&	-										& \textit{Myangat} 	\\
  $10^2$ 		& \textit{Hazarabam} 	& \textit{Zuut} 		\\
  $10^1$ 		& \textit{Dathabam} 	& \textit{Avaut} 		\\
\end{tabular}
\end{center}

\end{frame}

\subsection{Anarchiques}
\begin{frame}
\frametitle{test}
\end{frame}


\section{Stratégies et tactiques}
\frame{\sectionpage}


\section{Applications aux systèmes multi-agents}
\frame{\sectionpage}

\section{Bibliographie}
\bibliographystyle{plain}
\begin{frame}
\frametitle{Bibliographie}
\bibliography{../Bib.bib}
\end{frame}

\end{document}


