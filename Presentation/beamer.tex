
\PassOptionsToPackage{table}{xcolor}
\documentclass[utf8]{beamer}

\usepackage[T1]{fontenc}
\usepackage[french]{babel}
\usepackage{tikz}
\usepackage{bibentry}

\setbeamertemplate{navigation symbols}{}
\usetheme{Amsterdam}

\usetikzlibrary{calc}
\tikzstyle{item}=[rectangle,rounded corners=3pt,thick, dashed, color=beamer@blendedred, fill=gray!20]

% Romain
\newcommand{\cRM}[1]{\MakeUppercase{\romannumeral #1}}  % Capital
\newcommand{\cRm}[1]{\textsc{\romannumeral #1}} % Petit majuscule
\newcommand{\crm}[1]{\romannumeral #1}
% Siècle %
\newcommand{\siecle}[1]{\cRM{#1}\textsuperscript{e}~siècle}

\title{Les stratégies militaires dans les Systèmes Multi-Agents}
\author{Chloé Desdouits, William Dyce}
\date{\today}

%\AtBeginSection[]{
%  \begin{frame}[plain]
%  	\frametitle{Sommaire}
%  	\tableofcontents[currentsection]
%  \end{frame} 
%}

\begin{document}

\frame[plain]{\titlepage}

\begin{frame}[plain]
	\frametitle{Sommaire}
	\tableofcontents
\end{frame} 

\section{Organisations militaires}
\frame{\sectionpage}



\section{Stratégies et tactiques}
\frame{\sectionpage}

\begin{frame}
\pgfdeclareimage[width=6cm]{politique}{../ressources/alliances_ww1}
\pgfdeclareimage[width=7.5cm]{strategie}{../ressources/strategy_ww1}
\pgfdeclareimage[width=5cm]{tactique}{../ressources/Battles_of_Charleroi_ww1}

{\centering
\makebox[0pt]{%
\begin{tikzpicture}
\node[] (Si) at(0,-3) {\pgfbox[left,center] {\pgfuseimage{strategie}}};
\node[] (Pi) at(12.6,-2.66) {\pgfbox[right,bottom] {\pgfuseimage{politique}}};
\node[] (Ti) at(12.6,-5.7) {\pgfbox[right,center] {\pgfuseimage{tactique}}};
\node[draw,item] (P) at(5.5,0.4) {Politique};
\node[draw,item] (S) at(8.5,-3.2) {Stratégie};
\node[draw,item] (T) at(5.5,-6.5) {Tactique};
\end{tikzpicture}%
}\par
}
\footlineextra{\cite{ww1, military_strategy, tactic}}
\end{frame}


\subsection{Stratégies}

\begin{frame}
\frametitle{Sun Tzu}
\framesubtitle{Chine (\siecle{6} BC)}
\begin{tikzpicture}[remember picture,overlay]
	\node[anchor=north east,inner sep=0pt] at ($(current page.north east)+(-0.1cm,-0.75cm)$) {
	\includegraphics[width=1.9cm]{../ressources/sun_tzu_general}
  };
\end{tikzpicture}
\begin{quote}“L'art de la guerre, c'est de soumettre l'ennemi sans combat.”\end{quote}
\vfill
\begin{columns}[t]
\begin{column}{0.5\linewidth}
Préceptes
\begin{itemize}
\item prendre toutes les possessions de l'adversaire et les conserver intactes
\item adaptabilité, préparation, connaissance du terrain et des forces en présence (espionnage)
\end{itemize}
\end{column}
\begin{column}{0.5\linewidth}
Axes stratégiques
\begin{enumerate}
\item cause morale
\item conditions climatiques
\item conditions géographiques
\item qualités du dirigeant
\item organisation et discipline
\end{enumerate}
\end{column}
\end{columns}

\footlineextra{\cite{tzu1997art, sun_tzu_fighting, sun_tzu_wiki}}
\end{frame}

\begin{frame}{Alexandre le grand}
\framesubtitle{Grèce (\siecle{4} BC)}
\begin{tikzpicture}[remember picture,overlay]
	\node[anchor=north east,inner sep=0pt] at ($(current page.north east)+(-0.1cm,-0.75cm)$) {
	\includegraphics[trim=0cm 8cm 0cm 0cm, clip=true, width=1.9cm]{../ressources/AlexanderTheGreat_Bust}
  };
\end{tikzpicture}
\begin{quote}“Ce qui ne me tue pas me rend plus fort.”\end{quote}
\vfill
\begin{columns}[t]
\begin{column}{0.5\linewidth}
Préceptes
\begin{itemize}
\item conscription et intégration des peuples vaincus
\item allègement de l'équipement des troupes
\end{itemize}
\end{column}
\begin{column}{0.5\linewidth}
Axes stratégiques
\begin{enumerate}
\item assurer ses arrières
\item choisir judicieusement la voie d'accès pour chaque conquête
\end{enumerate}
\end{column}
\end{columns}
\footlineextra{\cite{alexander_the_great, alexandre_balkans}}
\end{frame}

\begin{frame}{Julius Caesar}
\framesubtitle{Italie (\siecle{1} BC)}
\begin{tikzpicture}[remember picture,overlay]
	\node[anchor=north east,inner sep=0pt] at ($(current page.north east)+(-0.1cm,-0.75cm)$) {
	\includegraphics[trim=0cm 8cm 0cm 0cm, clip=true, width=1.9cm]{../ressources/AlexanderTheGreat_Bust}
  };
\end{tikzpicture}
\begin{quote}“.”\end{quote}
\vfill
\begin{columns}[t]
\begin{column}{0.5\linewidth}
Préceptes
\begin{itemize}
\item .
\end{itemize}
\end{column}
\begin{column}{0.5\linewidth}
Axes stratégiques
\begin{enumerate}
\item .
\end{enumerate}
\end{column}
\end{columns}
\footlineextra{\cite{caesar_wiki}}
\end{frame}

\begin{frame}{Napoléon Bonaparte}
\framesubtitle{France (\siecle{18})}
\begin{tikzpicture}[remember picture,overlay]
	\node[anchor=north east,inner sep=0pt] at ($(current page.north east)+(-0.1cm,-0.75cm)$) {
	\includegraphics[trim=0cm 2cm 0cm 0cm, clip=true, width=1.9cm]{../ressources/napoleon}
  };
\end{tikzpicture}
\begin{quote}“Réunir ses feux contre un seul point ; une fois la brèche faite, l’équilibre est rompu, tout le reste devient inutile.”\end{quote}
\vfill
\begin{columns}[t]
\begin{column}{0.5\linewidth}
Préceptes
\begin{itemize}
\item recherche systématique de la bataille
\item destruction totale des forces adverses
\item être le plus fort à l’endroit où l’on a décidé de frapper le coup décisif
\end{itemize}
\end{column}
\begin{column}{0.5\linewidth}
Axes stratégiques
\begin{enumerate}
\item vitesse de manœuvre : {\emph Blitzkrieg}
\item fortifications
\item lignes de réapprovisionnement provisoires
\end{enumerate}
\end{column}
\end{columns}

\footlineextra{\cite{napoleon, napoleon_wiki, napoleon_portrait}}
\end{frame}


\section{Applications aux systèmes multi-agents}
\frame{\sectionpage}

\section{Bibliographie}
\bibliographystyle{plain}
%\bibliographystyle{amsalpha}
\begin{frame}[allowframebreaks]
\frametitle{Bibliographie}
\bibliography{../Bib.bib}
\end{frame}

\end{document}


